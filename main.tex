\documentclass{article}
\usepackage[utf8]{inputenc}

\title{scientific writing: introduction assignment}
\author{Torr Polakow}
\date{November 2018}

\begin{document}

\maketitle

\section{Introduction}

People tend to make irrational decisions . Irrational behavior is defined here as any behavior that reflects a violation of basic laws that stem from classical probability theory . In this paper, we focus on the conjunction and disjunction fallacies, which violate the law of total probability:
The conjunction fallacy occurs when a person judges the probability of the conjunction of two events to be more likely than either of the constituent events. 
The disjunction fallacy occurs when a person judges the probability of the disjunction of two events to be less likely than either of the constituent events.

To account for these fallacies, the emerging field of Quantum Cognition takes methods and concepts from quantum probability theory and uses them to explain and model decision-making findings. The hypothesis behind quantum cognition models is that irrational behavior obeys the laws of quantum probability theory rather than classical probability theory.

One of the basic principles that ushers the field of Quantum Cognition was the violation of the unicity principle.
This assumption is broken as soon as we allow incompatible questions into the theory, which cause measurements to be non-commutative. Incompatible questions cannot be evaluated on the same basis, so they require setting up separate sample spaces. This leads to conjunction and disjunction fallacies.
Quantum probability does not assume the principle of unicity, thus allowing one to use a partial Boolean algebra; each set of questions can be answered using one sample space in a Boolean fashion. 
All Boolean sub-algebras are pasted together in a coherent but non-Boolean fashion.

Furthermore, previous work has shown that quantum probability (QP) can be used to model cognitive fallacies, specifically, conjunction and disjunction. Two concepts analyzed in the fallacies lie in the same Hilbert space and represent two different reference frames. This framework can account for the irrationalities but not for the rational behavior.
Therefore, a simple quantum model of a word association system was proposed, wherein each concept was represented by a {\em different} Hilbert space. This quantum model of word entanglement can account for how we interpret semantic phrases as opposed to each of its conjugates.

Recently, debated the Quantum Cognition models. Their main argument was that the model can explain the conjunction fallacy but fails to account for double conjunction. They showed that an un-negligible percent of the participants ($~10\%$) is  making this fallacy. Thus, these results suggest that people's probability estimates do not follow quantum probability theory.

We propose a quantum model based on multi-qubits states, where each concept is represented by a different Hilbert space, i.e. qubit. 
The model can account for rational and irrational behaviors. 
In contrast, previous models have not used multi-qubit states for judgment decisions and did not account for rational an irrational decisions simultaneously.
Our model results in novel bounds on the degree of irrationality one can express, both in the conjunction and the disjunction scenarios.

To test our model's predictions, an online questionnaire containing several instantiations of conjunction and disjunction fallacies' scenarios was administered.
Using our model, we first show that it describes {\em all} participant results, i.e., both rational and irrational, for all the questions. 
We then show that the questionnaire's data, as well as data from, uphold our model's bound predictions.

\end{document}
